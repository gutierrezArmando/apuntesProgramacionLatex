%!TEX root = P_Manual.tex
\usepackage[spanish]{babel}

\usepackage{graphicx,hyperref,url, materialbeamer}
\usepackage{braket}
%\usepackage{euler}
\usepackage{listings}
\usepackage{booktabs}% to use \toprule
\usepackage{xfrac}% to use \sfrac{}{}
\usepackage{ragged2e} %to use \justify

\graphicspath{ {./figs/} }
\setbeamercovered{transparent}
\lstdefinestyle{customc}{
  belowcaptionskip=1\baselineskip,
  breaklines=true,
  xleftmargin=\parindent,
  language=C,
  showstringspaces=false,
  basicstyle=\footnotesize\ttfamily,
  keywordstyle=\bfseries\color{green!40!black},
  commentstyle=\itshape\color{purple!40!black},
  identifierstyle=\color{blue},
  stringstyle=\color{orange},
  extendedchars=true,
  breakatwhitespace=false
}
\lstset{escapechar=@,style=customc}



\usefonttheme{professionalfonts} % using non standard fonts for beamer
%\usefonttheme{serif}

% The title of the presentation:
%  - first a short version which is visible at the bottom of each slide;
%  - second the full title shown on the title slide;
\title[Programación]{Programación}

% Optional: a subtitle to be dispalyed on the title slide
\subtitle{Tronco Común de Ingeniería \\ FCQI}

% The author(s) of the presentation:
%  - again first a short version to be displayed at the bottom;
%  - next the full list of authors, which may include contact information;
\author[Violeta Ocegueda]{Violeta Ocegueda} 
  
%\titlegraphic{\includegraphics[width=\textwidth]{atac-logo}}

% The institute:
%  - to start the name of the university as displayed on the top of each slide
%    this can be adjusted such that you can also create a Dutch version
%  - next the institute information as displayed on the title slide
\institute[UABC]{Profesor-Investigador \\ Facultad de Ciencias Químicas e Ingeniería \\ Universidad Autónoma de Baja California \\ Campus Tijuana}

% Add a date and possibly the name of the event to the slides
%  - again first a short version to be shown at the bottom of each slide
%  - second the full date and event name for the title slide
\date[\today]{\today}

\providecommand{\di}{\mathop{}\!\mathrm{d}}
\providecommand*{\der}[3][]{\frac{d\if?#1?\else^{#1}\fi#2}{d #3\if?#1?\else^{#1}\fi}} 
 \providecommand*{\pder}[3][]{% 
    \frac{\partial\if?#1?\else^{#1}\fi#2}{\partial #3\if?#1?\else^{#1}\fi}% 
  }