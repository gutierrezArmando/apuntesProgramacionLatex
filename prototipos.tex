\section*{Prototipos, llamada y cuerpo de la función}

%- - - - - - - - - - - - - - - - - Título - - - - - - - - - - - - - - - - - -%
\begin{frame}[c] 
\centering
\huge \textbf{Prototipos, llamada y cuerpo de la función}
\end{frame}



% - - - - - - - - - - - - - - - - - Slide 01 - - - - - - - - - - - - - - - -
\begin{frame}[t]
    \frametitle{Prototipos, llamada y cuerpo de la función}
    \begin{center}
        \textbf{Prototipo}
    \end{center}
    \vspace{-6mm}
    \hspace{5mm}Un prototipo es el encabezado de una función, es decir la primera línea de la función. La ventaja de utilizar prototipos es que las funciones pueden estar en cualquier lugar y en cualquier orden y pueden ser llamadas desde cualquier punto del programa principal o de otra función.
    \vspace{-2mm}
    \begin{block}{Orden}
        \begin{enumerate}
            \item Declaración de los prototipos de cada función.\pause
            \item Cuerpo de la función principal.\pause
            \item Al final el cuerpo de cada función.
        \end{enumerate}
    \end{block}
\end{frame}



% - - - - - - - - - - - - - - - - - Slide 02 - - - - - - - - - - - - - - - -
\begin{frame}[fragile]
    \frametitle{Prototipos, llamada y cuerpo de la función}
    \begin{center}
        \textbf{EJEMPLO}
    \end{center}
    \begin{lstlisting}[basicstyle=\ttfamily\tiny]
#librearias
#Constantes
funcionA();/*Prototípo de la función A*/
funcionB();/*Prototípo de la función B*/
int main(){
    variables
    Cuerpo del programa principal
    ............
    funcionaA();
    funcionaB();
    .......
    return 0;
}
funcionA(){
    variables locales de función
    instrucciones
}
funcionB(){
    variables locales de función
    instrucciones
}
\end{lstlisting}
\end{frame}
