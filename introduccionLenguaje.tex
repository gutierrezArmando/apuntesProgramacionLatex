%!TEX root = P_Manual.tex
\setlength{\parskip}{\baselineskip} 
\section*{Lenguaje}

% --------------------------------- Slide 01 ----------------------------------------- %
\begin{frame}[c] 
\frametitle{}
\centering
\huge \textbf{Introducción al lenguaje de programación}
\end{frame}


% --------------------------------- Slide 02 ----------------------------------------- %
\begin{frame}[t, fragile]
\frametitle{Estructura básica de un programa}
\begin{lstlisting}
/* Inclusion de librerias */

void main() /* cabecera de funcion */
{ /* inicio de la funcion main */

... /* Sentencias */

} /* fin de la funcion main */
\end{lstlisting}
\end{frame}


% --------------------------------- Slide 03 ----------------------------------------- %
\begin{frame}[fragile]
\frametitle{Estructura básica de un programa para la clase}
\begin{lstlisting}
/* Inclusion de librerias */

void main() /* cabecera de funcion */
{ /* inicio de la funcion main */

/* Declaracion de variables */

/* Entrada de datos */

/* Procesamiento */

/* Impresion de resultados */

} /* fin de la funcion main */
\end{lstlisting}
\end{frame}


% --------------------------------- Slide 04 ----------------------------------------- %
\begin{frame}[fragile, t]
\frametitle{Zonas de memoria}
\textbf{Variables}\\
\footnotesize
\begin{itemize}
	\item Son objetos que pueden cambiar su valor durante la ejecución de un programa.
	\item En C una \textit{variable} es una posición en memoria con nombre donde se almacena un valor de un cierto tipo de dato.
\end{itemize}
\vspace{-2mm}
La \textit{declaración} de una variable es una sentencia que proporciona información de la variable al compilador C. Su sintaxis es:
\begin{lstlisting}
<tipo_variable> <nombre_variable> = <valor_inicial>;
\end{lstlisting}
\vspace{-4mm}
Donde:
\vspace{-4mm}
\begin{itemize}
	\item \textit{tipo\_variable} es el nombre de un tipo de dato conocido por C.
	\item \textit{nombre\_variable} es un identificador (nombre) válido en C.
	\item \textit{valor\_inicial} es el valor de inicialización de la variable.
\end{itemize}
\end{frame}

% --------------------------------- Slide 04 ----------------------------------------- %