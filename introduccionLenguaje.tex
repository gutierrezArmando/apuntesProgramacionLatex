%!TEX root = P_Manual.tex
\setlength{\parskip}{\baselineskip} 
\section{Lenguaje}

% ----------------------- Slide 01-------------------------------- %
\begin{frame}[c] 
\frametitle{}
\centering
\huge \textbf{Introducción al lenguaje de programación}
\end{frame}


% --------------------------- Slide 02 ---------------------------------- %
\begin{frame}[t, fragile]{Estructura básica de un programa}
%	\frametitle{Estructura básica de un programa}
	\begin{lstlisting}
/* Inclusion de librerias */

void main() /* cabecera de funcion */
{ /* inicio de la funcion main */

... /* Sentencias */

} /* fin de la funcion main */
\end{lstlisting}
\end{frame}


% ------------------------- Slide 03 --------------------------------- %
\begin{frame}[fragile]{Estructura básica de un programa para la clase}
%	\frametitle{Estructura básica de un programa para la clase}
	\begin{lstlisting}
/* Inclusion de librerias */

void main() /* cabecera de funcion */
{ /* inicio de la funcion main */

/* Declaracion de variables */

/* Entrada de datos */

/* Procesamiento */

/* Impresion de resultados */

} /* fin de la funcion main */
\end{lstlisting}
\end{frame}


% -------------------------- Slide 04 ---------------------------------- %
\begin{frame}[fragile, t]{Zonas de memoria}
%	\frametitle{Zonas de memoria}
	\textbf{Variables}\\
	\footnotesize
	\begin{itemize}
		\item Son objetos que pueden cambiar su valor durante la ejecución de un programa.
		\item En C una \textit{variable} es una posición en memoria con nombre donde se almacena un valor de un cierto tipo de dato.
	\end{itemize}
	\vspace{-2mm}
	La \textit{declaración} de una variable es una sentencia que proporciona información de la variable al compilador C. Su sintaxis es:
	\begin{lstlisting}
<tipo_variable> <nombre_variable> = <valor_inicial>;
\end{lstlisting}
	\vspace{-4mm}
	Donde:
	\vspace{-4mm}
	\begin{itemize}
		\item \textit{tipo\_variable} es el nombre de un tipo de dato conocido por C.
		\item \textit{nombre\_variable} es un identificador (nombre) válido en C.
		\item \textit{valor\_inicial} es el valor de inicialización de la variable.
	\end{itemize}
\end{frame}

% --------------------------- Slide 05 --------------------------------- %
\begin{frame}[t, fragile]{Zonas de memoria}
	%\frametitle{Zonas de memoria}
	\textbf{Tipos de datos}\\
	\footnotesize
	Los tipos de datos básicos son:
	\begin{itemize}
		\item enteros
		\item flotantes
		\item caracteres
	\end{itemize}
	\textbf{Constantes}\\
	Las \textit{constantes} son datos que no cambian durante la ejecución de un programa.
	\begin{lstlisting}
#define NUEVALINEA \n 
#define PI 3.141592 
#define VALOR 54
\end{lstlisting}
\end{frame}


% --------------------------- Slide 06 --------------------------------- %
\begin{frame}[t]{Operadores}
\textbf{Operadores de asignación y expresión}\\
\begin{center}
	\begin{tabular}{ccc}
		\toprule
		\textbf{Operador} & \textbf{Expresión} & \textbf{Explicación} \\
		\midrule
		+= & c += 7 & c = c + 7 \\ \hline
		-= & d -= 4 & d = d - 4 \\ \hline
		*= & e *= 5 & e = e * 5 \\ \hline
		/= & f/= 3 & f = f / 3 \\ \hline
		\hspace{1mm} \%= & g \%= 9 & g = g \% 9 \\
		\bottomrule
	\end{tabular}
\end{center}
\end{frame}

% --------------------------- Slide 07 --------------------------------- %
\begin{frame}[t]{Operadores}
\textbf{Operadores aritméticos}\\
\vspace{5mm}
\small
\begin{center}
	\begin{tabular}{cccc}
		\toprule
		\textbf{Operador} & \textbf{Operación} & \textbf{Expresión algebraica} & \textbf{Expresión en C}\\
		\midrule
		+ & Suma & f + 7 & f + 7 \\ \hline
		- & Resta & p - c & p - c\\ \hline
		* & Multiplicación & bm & b * m\\ \hline
		/ & División & $\frac{x}{y}$ ó $\sfrac{x}{y}$ & x / y \\ \hline
	\end{tabular}
\end{center}
\end{frame}

% --------------------------- Slide 08 --------------------------------- %
\begin{frame}[t]{Operadores}
\textbf{Operadores de relación}\\ \vspace{5mm}
\begin{center}
\begin{tabular}{ccl}
	\toprule
	\textbf{Operador} & \textbf{Ejemplo} & \textbf{Significado}\\
	\midrule
	== & x == y & x es igual que y \\ \hline
	!= & x != y & x es diferente de y \\ \hline
	> & x > y & x es mayor que y \\ \hline
	< & x < y & x es menor que y \\ \hline
	>= & x >= y & x es mayor o igual que y \\ \hline
	<= & x <= y & x es menos o igual que y \\ 
	\bottomrule
\end{tabular}
\end{center}
\end{frame}

% --------------------------- Slide 09 --------------------------------- %
\begin{frame}[t]{Operadores}
\textbf{Operadores lógicos: AND lógico}\\ \vspace{3mm}
\small
\begin{center}
%\caption{Tabla de verdad para el operador \textit{\&\&} (AND l\'ogico).}
\begin{tabular}{llc}
	\toprule
	\textbf{expresión} & \textbf{expresión2} & \textbf{expresión1 \&\& expresión2}\\
	\midrule \hline
	0 & 0 & 0 \\
	0 & diferente de cero & 0\\
	diferente de cero & 0 & 0\\
	diferente de cero & diferente de cero & 1\\ \hline
	\bottomrule
\end{tabular}
\end{center}
\end{frame}

% --------------------------- Slide 10 ------------------------------ %
\begin{frame}[t]{Operadores}
\textbf{Operadores lógicos: OR lógico}\\ \vspace{5mm}
\centering
%\caption{Tabla de verdad para el operador \textit{| |} (OR l\'ogico).}
\begin{tabular}{llc}
	\toprule
	\textbf{expresión1} & \textbf{expresión2} & \textbf{expresión1 $| |$ expresión2}\\
	\midrule \hline
	0 & 0 & 0 \\
	0 & diferente de cero & 1\\
	diferente de cero & 0 & 1\\
	diferente de cero & diferente de cero & 1\\ \hline
	\bottomrule
\end{tabular}
\end{frame}

% --------------------------- Slide 11 --------------------------------- %
\begin{frame}[t]
\frametitle{Operadores}
\textbf{Operadores lógicos: NOT lógico}\\ \vspace{5mm}
\centering
%\caption{Tabla de verdad para el operador \textbf{!} (negaci\'on l\'ogico).}
\begin{tabular}{lc}
\toprule
\textbf{expresión} & \textbf{!expresión}\\
\midrule \hline
0 & 1 \\
diferente de cero & 0\\
\bottomrule
\end{tabular}
\end{frame}

% --------------------------- Slide 12 --------------------------------- %
\begin{frame}[t]{Operadores}
\textbf{Operadores de incremento y decremento}\\ \vspace{5mm}
\footnotesize
\begin{center}
%\caption{Operadores de incremento y decremento}
\begin{tabular}{ccp{7.5cm}}
	\toprule
	\textbf{Operador} & \textbf{Ejemplo} & \textbf{Explicación}\\
	\midrule
	++&++a& Incrementar a en 1 y después utiliza el nuevo valor de a en la expresión en la que reside.\\
	++ & a++ & Utiliza el valor actual de a en la expresión en la que reside, y después la incrementa en 1.\\
	$--$ & $--$b & Decrementar b en 1 y después utiliza el nuevo valor de b en la expresión en la que reside.\\
	$--$ & b$--$ & Utiliza el valor actual de b en la expresión en la que reside, y después decrementa b en 1.\\
	\bottomrule
\end{tabular}
\end{center}
\end{frame}


% --------------------------- Slide 13 --------------------------------- %
\begin{frame}[t]{Operadores}
\textbf{Jerarquía de operadores}\\ \vspace{5mm}
\scriptsize
%\begin{table}[h]
\centering
%\caption{Jerarqu\'ia de operadores}
\begin{tabular}{ccll}
	\toprule
	\textbf{Jerarqu\'ia} & \textbf{Operadores} & \textbf{Asociatividad} & \textbf{Tipo}\\ 
	\midrule 
	Mayor & $++$ $--$ $+$ $-$ $!$ & derecha a izquierda & unario\\ \cline{2-4}
	| & * / \% & izquierda a derecha & multiplicativo\\ \cline{2-4}
	| & + $-$ & izquierda a derecha & aditivo\\ \cline{2-4}
	| & $<$ $<=$ $>$ $>=$ & izquierda a derecha & de relaci\'on \\ \cline{2-4}
	| & == != & izquierda a derecha & de relación\\ \cline{2-4}
	| & \&\& & izquierda a derecha & AND lógico\\ \cline{2-4}
	| & $| |$ & izquierda a derecha & OR lógico \\ \cline{2-4}
	Menor & $=$, $+=$, $-=$, $*=$, $/=$, $\%=$ & derecha a izquierda & de asignación \\ 
	\bottomrule
\end{tabular}
%\end{table}
\end{frame}


% --------------------------- Slide 14 --------------------------------- %
\begin{frame}{Instrucción de salida}
La instrucción de salida utilizada en C se denomina \textbf{printf}. Su sintaxis es:\\
\vspace{5mm}

{\small \textit{printf( ``texto a imprimir'' );}\\
\textit{printf( ``texto a imprimir \%formato\_de\_impresión'',variable\_a\_imprimir );}}
\end{frame}


% --------------------------- Slide 15 --------------------------------- %
\begin{frame}[t]
\frametitle{Formato de salida con printf}
\textbf{Especificadores de conversión entera para \textit{printf}}\\ \vspace{5mm}
\scriptsize
%\begin{table}[h]
\begin{center}
%\caption{Especificadores de conversi\'on entera para \textit{printf}}
\begin{tabular}{lp{7.5cm}}
	\toprule
	\textbf{Especificador} & \textbf{Descripción}\\
	\midrule
	\textbf{\%d} & Despliega un entero decimal con signo.\\ 
	\textbf{\%i} & Despliega un entero decimal con signo. [Nota: los especificadores i y d son diferentes cuando se utilizan con \textbf{scanf}.]\\ 
	\textbf{\%o} & Despliega un entero octal sin signo.\\
	\textbf{\%u} & Despliega un entero decimal sin signo. \\
	\textbf{\%x} ó \textbf{\%X} & Despliega un entero hexadecimal sin signo. \textbf{X} provoca que se desplieguen los dígitos de \textbf{0} a \textbf{9} y las letras de \textbf{A} a \textbf{F}, y \textbf{x} provoca que se desplieguen los dígitos de \textbf{0} a \textbf{9} y las letras de \textbf{a} a \textbf{f}.\\ 
	\textbf{h} ó \textbf{l} (letra l) & Se coloca antes de cualquier especificador de conversión entera para indicar que se despliega un entero corto o largo, respectivamente. Las letras \textbf{h} y \textbf{l} son llamadas con más precisión \textit{modificadores de longitud}.\\
	\bottomrule
\end{tabular}
%\end{table}
\end{center}
\end{frame}
% --------------------------- Slide 16 --------------------------------- %
\begin{frame}[t]{Formato de salida con printf}
\textbf{Especificadores de conversi\'on de punto flotante para \textit{printf}}\\ \vspace{5mm}
\scriptsize
\centering
%\caption{Especificadores de conversi\'on de punto flotante para \textit{printf}}
\begin{tabular}{lp{7.5cm}}
	\toprule
	\textbf{Especificador} & \textbf{Descripción}\\
	\midrule 
	\textbf{\%e} \'o \textbf{\%E} & Despliega un valor de punto flotante con notaci\'on exponencial.\\
	\textbf{\%f} & Despliega un valor de punto flotante con notaci\'on de punto fijo.\\
	\textbf{\%g} \'o \textbf{\%G} & Despliega un valor de punto flotante con el formato de punto flotante \textbf{f}, o con el formato exponencial \textbf{e} (o \textbf{E}) basado en la magnitud del valor. \\
	\textbf{L} & Se coloca antes del especificador de conversión para indicar que se desplegar\'a un valor de punto flotante \textbf{long double}\\ 
	\bottomrule
\end{tabular}
\end{frame}

% --------------------------- Slide 17 --------------------------------- %
\begin{frame}[t]{Formato de salida con printf}
\textbf{Especificadores de conversión de caracteres y cadenas para \textit{printf}}\\ \vspace{5mm}
%\scriptsize
\centering
%\caption{Especificadores de conversi\'on de caracteres y cadenas para \textit{printf}}
\begin{tabular}{lp{7.5cm}}
	\toprule
	\textbf{Especificador} & \textbf{Descripci\'on}\\
	\midrule
	\textbf{\%c} & Despliega caracteres individuales.\\
	\textbf{\%s} & Despliega cadenas de caracteres. \\
	\bottomrule
\end{tabular}
\end{frame}

% --------------------------- Slide 18 --------------------------------- %
\begin{frame}[t]{Formato de salida con printf}
\textbf{Otros especificadores de conversi\'on para \textit{printf}}\\ \vspace{5mm}
\scriptsize
\begin{center}
%\caption{Otros especificadores de conversi\'on para \textit{printf}}
\begin{tabular}{lp{7.5cm}}
	\toprule
	\textbf{Especificador} & \textbf{Descripción}\\
	\midrule
	\textbf{\%p} & Despliega un valor apuntador de manera definida por la implementación.\\ 
	\textbf{\%n} & Almacena el número de caracteres ya desplegados en la instrucci\'on \textbf{printf} actual. Proporciona un apuntador a un entero como el argumento correspondiente. No despliega valor alguno.\\ 
	\textbf{\%\%} & Despliega el caracter de porcentaje.\\
	\bottomrule
\end{tabular}
\end{center}
\end{frame}

% --------------------------- Slide 19 --------------------------------- %
\begin{frame}[t]{Secuencias de escape}
\small
\begin{center}
\begin{tabular}{p{3.5cm}p{6.5cm}}
	\toprule
	\textbf{Secuencia de escape} & \textbf{Descripción}\\
	\midrule
	\textbackslash a (alerta o campana) & Provoca una alerta sonora (campana) o una alerta visual.\\
	\textbackslash \textbackslash (diagonal invertida) & Despliega el caracter de diagonal invertida (\textbackslash).\\
	\textbackslash ' (comilla sencilla) & Despliega el caracter de comilla sencilla (').
	\\
	\textbackslash '' (comilla doble) & Despliega el caracter de comilla doble (''). \\ 
	\textbackslash ? (interrogación) & Despliega el caracter de interrogación (?).\\
	\textbackslash n (nueva línea) & Mueve el cursor al inicio de la siguiente línea.\\
	\bottomrule
\end{tabular}
\end{center}
\end{frame}

% --------------------------- Slide 20 --------------------------------- %
\begin{frame}[t]{Secuencias de escape (Continuación)}
\vspace{-4mm}
\small
\begin{center}
	\begin{tabular}{p{4cm}p{6cm}}
		\toprule
		\textbf{Secuencia de escape} & \textbf{Descripción}\\
		\midrule 
		\textbackslash t (tabulador horizontal) & Mueve el cursor a la siguiente posición del tabulador.\\
		\textbackslash b (retroceso) & Mueve el cursor una posición hacia atrás en la línea actual.\\
		\textbackslash f (nueva página o avance de página) & Mueve el cursor al inicio de la siguiente página l\'ogica.\\
		\textbackslash r (retorno de carro) & Mueve el cursor al principio de la línea actual.\\
		\textbackslash v (tabulador vertical) & Mueve el cursor a la siguiente posición del tabulador vertical.\\
		\bottomrule
	\end{tabular}
\end{center}
\end{frame}

% --------------------------- Slide 21 --------------------------------- %
\begin{frame}{Banderas de la cadena de control de formato}
\small
%\begin{table}[h]
\centering
%\caption{Banderas de la cadena de control de formato}
\begin{tabular}{lp{6cm}}
	\toprule
	\textbf{Bandera} & \textbf{Descripción}\\
	\midrule 
	\textbackslash$-$ (signo menos) & Justifica la salida a la izquierda dentro del campo especificado.\\ 
	\hline
	\textbackslash+ (signo más) & Despliega el signo más que precede a los valores positivos, y un signo menos que precede a los valores negativos.\\
	\hline
	\textit{espacio} & Imprime un espacio antes de un valor positivo no impreso con la bandera +.\\
	\bottomrule
\end{tabular}
%\end{table}
\end{frame}

% --------------------------- Slide 22 --------------------------------- %
\begin{frame}[t]{Instrucción de entrada}
La instrucción de entrada utilizada en C se denomina \textbf{scanf}. Su sintaxis es:\\
\textit{scanf( ``especificador\_de\_conversión'',\&nombre\_variable );}\\
Donde:
%----------LIST----------
\begin{itemize}
	\item \textbf{especificador\_de\_conversión} describe el formato de los datos de entrada.
	\item \textbf{\&} asigna los datos, en el formato especificado, a la variable especificada.
	\item \textbf{nombre\_variable} variable a la cual se le asigna los datos de entrada.
\end{itemize}
\end{frame}

% --------------------------- Slide 23 --------------------------------- %
\begin{frame}[t]{Formato de entrada con scanf}
\textbf{Especificadores de conversi\'on de enteros para \textit{scanf}}
\vspace{-2mm}
\small
\begin{center}
%\caption{Especificadores de conversi\'on de enteros para \textit{scanf}.}
\begin{tabular}{lp{6cm}}
	\toprule
	\textbf{Especificador} & \textbf{Descripci\'on}\\
	\midrule
	\%d & Lee un entero decimal con signo (el signo es opcional). El argumento correspondiente es un apuntador a un entero. \\
	\%i & Lee un entero decimal, octal, o hexadecimal con signo (opcional). El argumento correspondiente es un apuntador a un entero.\\
	\%o & Lee un entero octal. El argumento corespondiente es un apuntador a un entero sin signo.\\
	\bottomrule
\end{tabular}
\end{center}
\end{frame}

% --------------------------- Slide 24 --------------------------------- %
\begin{frame}[t]{Formato de entrada con scanf (Continuación)}
\textbf{Especificadores de conversi\'on de enteros para \textit{scanf}}
\vspace{-2mm}
\small
\begin{center}
	%\caption{Especificadores de conversi\'on de enteros para \textit{scanf}.}
	\begin{tabular}{lp{6cm}}
		\toprule
		\textbf{Especificador} & \textbf{Descripci\'on}\\
		\midrule
		\%u & Lee un entero decimal sin signo. El argumento correspondiente es un apuntador a un entero sin signo.\\
		\%x o \%X & Lee un entero hexadecimal. El argumento correspondiente es un apuntador a un entero sin signo.\\
		h \'o l & Se coloca antes de cualquier especificador de conversi\'on, para indicar que se introducir\'a un entero corto o largo, respectivamente.\\
		\bottomrule
	\end{tabular}
\end{center}
\end{frame}

% --------------------------- Slide 25 --------------------------------- %
\begin{frame}[t]
\frametitle{Formato de entrada con scanf}
\textbf{Especificadores de conversión de números de punto flotante para \textit{scanf}}\vspace{5mm}
\scriptsize
%\begin{table}[h]
\centering
%\caption{Especificadores de conversi\'on de n\'umeros de punto flotante para \textit{scanf}.}
\begin{tabular}{lp{6cm}}
	\toprule
	\textbf{Especificador} & \textbf{Descripci\'on}\\
	\midrule
	\%e, \%E, \%f, \%g \'o \%G & Lee un valor de punto flotante. El argumento correspondiente es un apuntador a un valor de punto flotante.\\
	l \'o L & Se coloca antes de cualquier especificador de conversión para indicar que se introducir\'a un valor \textbf{double} o \textbf{long double}. El argumento correspondiente es un apuntador a una varible \textbf{double} o \textbf{long double}.\\ 
	\bottomrule
\end{tabular}
\end{frame}

% --------------------------- Slide 26 --------------------------------- %
\begin{frame}[t]
\frametitle{Formato de entrada con scanf}
\textbf{Especificadores de conversión de cadenas y caracteres para \textit{scanf}}\vspace{5mm}\\
\scriptsize
\centering
\begin{tabular}{lp{6cm}}
	\toprule
	\textbf{Especificador} & \textbf{Descripción}\\
	\midrule
	\%c & Lee un caracter. El argumento correspondiente es un apuntador a \textbf{char}; no agrega el caracter nulo ('\textbackslash0').\\
	\%s & Lee una cadena. El argumento correspondiente es un apuntador a un arreglo de tipo \textbf{char} que sea lo suficientemente grande para almacenar la cadena y el caracter nulo ('\textbackslash0'), el cual se agrega automáticamente. \\
	\bottomrule
\end{tabular}
\end{frame}