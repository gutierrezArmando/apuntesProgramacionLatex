\section*{Funciones Sencillas}

%- - - - - - - - - - - - - - - - - Título - - - - - - - - - - - - - - - - - -%
\begin{frame}[c] 
\centering
\huge \textbf{Funciones Sencillas}
\end{frame}



% - - - - - - - - - - - - - - - - - Slide 01 - - - - - - - - - - - - - - - -
\begin{frame}[fragile]{Funciones Sencillas}
\hspace{5mm}Son aquellas que no reciben parámetros o valores, ya que éstos se solicitan dentro de la función, luego se realizan las instrucciones (cálculos u operaciones) y normalmente se imprime el resultado.\\Ejemplo:
\begin{lstlisting}
void nombreFuncion() {
    Declaracion de variables;
    Cuerpo de la funcion;
    .
    .
}
\end{lstlisting}
\end{frame}



% - - - - - - - - - - - - - - - - - Slide 01 - - - - - - - - - - - - - - - -
\begin{frame}[fragile]{Funciones Sencillas}
\begin{center}
    \textbf{Llamadas a funciones sin paso de parametros}
\end{center}
\begin{lstlisting}
    printf("%d", nombreFuncion());
    
    variable = nombreFuncion();
    
    if(nombreFuncion() > expresion)
\end{lstlisting}
\end{frame}